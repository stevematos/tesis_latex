\documentclass[conference]{IEEEtran}
\IEEEoverridecommandlockouts
% The preceding line is only needed to identify funding in the first footnote. If that is unneeded, please comment it out.
\usepackage{cite}
\usepackage{amsmath,amssymb,amsfonts}
\usepackage{algorithmic}
\usepackage{graphicx}
\usepackage{textcomp}
\usepackage{xcolor}
\usepackage[utf8]{inputenc}
\def\BibTeX{{\rm B\kern-.05em{\sc i\kern-.025em b}\\kern-.08em
    T\kern-.1667em\lower.7ex\hbox{E}\kern-.125emX}}
\begin{document}

\title{Modelo hibrido de predicción de éxito de las ventas para empresas B2B}


\author{\IEEEauthorblockN{1\textsuperscript{st} Matos Manguinuri,Steve Sader}
\IEEEauthorblockA{\textit{Facultad de Ingeniería de Sistemas e}\\
\textit{Informática} \\
\textit{Universidad Nacional Mayor de San}\\
\textit{Marcos}\\
Lima, Perú \\
steve.matos@unmsm.edu.pe}
\and
\IEEEauthorblockN{2\textsuperscript{nd} Calderón Vilca, Hugo David}
\IEEEauthorblockA{\textit{Facultad de Ingeniería de Sistemas e}\\
\textit{Informática} \\
\textit{Universidad Nacional Mayor de San}\\
\textit{Marcos}\\
Lima, Perú}
}

\maketitle

\begin{abstract}
This document is a model and instructions for \LaTeX.
This and the IEEEtran.cls file define the components of your paper [title, text, heads, etc.]. *CRITICAL: Do Not Use Symbols, Special Characters, Footnotes, 
or Math in Paper Title or Abstract.
\end{abstract}

\begin{IEEEkeywords}
component, formatting, style, styling, insert
\end{IEEEkeywords}

\section{Introduccion}
\subsection{¿Problemas con el CRM?}
Normalmente para los equipos de ventas reciben la implementación de un
CRM en su estación de trabajo de mala manera , como [17] lo afirma en el
artículo del periódico El Financiero .“ Ya que el equipo de ventas tenía tan
pocos incentivos para mantener actualizados los datos solicitados, la
calidad de estos en el sistema se volvió cada vez menos confiable durante el
siguiente año.”. Esto provoca que el área de ventas sea la menos satisfecha
con esta solución.
Muchas empresas sufren un síndrome crónico en el que sus pronósticos de
CRM se convierten en un registro histórico y no en una guía para el futuro
[18] , estos datos al final se desperdician y no se usan para futuras
predicciones de las próximas ventas.
En conclusión, siempre la captación de clientes ha sido un tema vital para
cualquier empresa y sobre todo si su área de marketing es algo vital para el
negocio. Entonces la aplicación de una IA al área de ventas y sobre todo si usa un
CRM, afectara a los ingresos de la empresa y también un ahorro enorme de horas
hombres al área de ventas.

\subsection{Aplicaciones de IA al área de ventas}
Si bien la mayoría de los especialistas en marketing son expertos en leer informes
y métricas estándar, pocos están capacitados para analizar datos de múltiples
fuentes con la suficiente profundidad para informar mejores decisiones de
negocios [19] , esto impide que sea difícil para el área de marketing puedan
interpretar fácilmente la gran cantidad de datos que sea vengan de varias fuentes y
da por consecuencia que el área de ventas es las más perjudicada con este decisión
puesto que ellos son la parte operatoria y provocara que su área tenga que captar
clientes ciegamente.
El acceso a la inteligencia de marketing puede ayudar a las organizaciones a tomar
decisiones más informadas sobre el gasto en publicidad para futuras campañas,
segmentación de audiencia, combinación óptima de canales, etc [19] . Se refiere
sobre todo a los conocimientos aplicando IA al área de marketing, esto repercute
en un ahorro enorme de los gastos de marketing y también un ahorro de horas
hombre del área de ventas.\\
\section{TRABAJO RELACIONADOS}

\subsection{Modelos hibridos para ventas}
Se usaron diferentes modelos híbridos como ARIMA + BD Neural Network [3] y
ARIMA + RNN [4] en el área de la predicción de ventas, además que en [24] se
comparó modelos híbridos contra otros modelos usados en los últimos años.
En técnicas de predicción de ventas , la investigación de [3] hallaron que la
precisión del modelo hibrido (ARIMA + BD Neural Network ) que se planteo tiene un
rango de precisión de 98 a 99% frente a [4] con su modelo hibrido ARIMA + RNN
que logro un 97.36 , pero [3] el rango de predicción es a nivel mensual y no diaria
como lo es [4] . También nos mencionan del modelo hibrido PDPF [24] , del cual
supera a otros modelos comunes en la predicción de ventas como ARIMA , RVM ,
PF y PD, y nos menciona que la fortaleza de este hibrido es la precisión a pesar de
la poca data que se tenga, comparando con otros modelos que necesitan mucha
data para dar una predicción aceptable.\\

Además, en [36] se recalca que los métodos estadísticos pueden no lograr un
resultado de predicción deseable. Por lo que uso de un ANN puede potencialmente
mejorar la precisión de la previsión. Por ende nos menciona que los modelos
híbridos(del cual se habló de otras investigaciones en el capítulo 2.1 de esta
investigación) logran un mejor rendimiento porque pueden aprovechar los métodos
tanto estadísticos como de un ANN. En [23] dan una serie de comparaciones entre
muchos modelos tanto híbridos como otros que también son muy aplicados en la
predicción de ventas el cual el modelo más efectivo es el de descomposición STL +
Snaive, ARIMA y XGBoost (de forma individual) , del cual se afirma en el review que
las técnicas de descomposición superan a las técnicas hibridas y las técnicas
hibridas superan a los modelos individuales para la predicción de ventas.
Y por ultimo en [49] , proponen un hibrido poco común del cual es el modelo de Bass
/ Norton y el análisis de sentimientos, y es uno de los pocas investigaciones que
considera los sentimientos expresados en el contenido de las revisiones en línea
para la predicción de ventas, tanto del producto como en general.

\subsection{Técnicas de predicción de ventas}
En distintas investigación se usaron varias técnicas de deep learning para la
predicción de ventas [1] [2] y también nuevos modelos [5] [6] [7] .
En [1] se aplica deep learning a través de un DNN y se usa las variables más que
todo del producto para poder predecir la cantidad de productos que se venderá de
forma individual de acuerdo a 10 variables descritas en la misma investigación [1] .
También [2] usa otra técnica de deep learning , el cual es PRNN , y mencionan que
esta técnica supera a técnicas como SARIMA y ETS pero no a otros como MR ni E ,
por ende es un alternativa viable pero aún falta su aplicación en un gran conjunto de
series de tiempo de ventas.
En cambio en la predicción de ventas [5] usaron el método multifuncional Holt-
Winters y el método aditivo Holt-Winters pero en un modelo de predicción de ventas
mensual usando data de un sistema ERP a diferencia de [6] que propone un nuevo
modelo llamado TADA que consta de un codificador LSTM basado en tareas
múltiples y el decodificador LSTM basado en la atención dual y lo puede predecir en
diferentes series de tiempo , ademas que usa dos dataset ( OSW y Favorita ) para
poder contractar los resultados de la aplicación del modelo en diferentes contexto lo
cual supera en aspecto de cantidad y variedad de data a [5]

\subsection{Técnicas de predicción de clientes}

En distintas investigación se usaron varias técnicas de deep learning para la
predicción de clientes [11] [12] y también nuevos modelos [13] [14] ,o el uso de
tecnicas y modelos en un contexto nuevo [46] .
En [11] se usó las variables de los clientes junto con el clima para ver el flujo de
clientes a diferencia [12] que aplica más a las variables de los POS además que a
diferencia de otras investigaciones se usó variables de regularizaciones para poder
corregir.\\

Para el problema de la predicción de flujo de clientes , En [13] se aplico el uso de
clasificadores lineales y clasificadores basados pero solo usa el historial de
comportamiento de cliente a diferencia de [14] que comenta que no solo los tiempos
de clic, el tiempo de duración son los únicos factores que pueden demostrar las
preferencias de los usuarios y es más estos son solo una parte de los factores que
pueden demostrar las preferencias de los usuarios porque los factores que mas
influyen en la compra de los usuarios dependen de los caracteres del propio usuario
y del artículo por ende aplicaron Baying multinomial naive (MNB) con estos dos
grandes conjuntos de características superando a la investigación de [13] en este
aspecto.\\
En [46] se propone el modelo de Bag-of-words , el cual es un nuevo modelo para
este contexto, pero también nos menciona que lo usan en conjunto con proyecciones
aleatorias para el tema de la curse-of-dimensional que aflige a los modelos bag-of-
words. Aunque es modelo nuevo para este contexto, es muy competente en el
problema de predicción de ventas.

\section{METODOLOGÍA}
La data será la base de datos de SEMINARIUM PERU S.A.C que alberga 939
tablas. De lo cual se estima que se obtendrá más de 6000 datos con una cantidad de
atributos importantes que se puede escoger para el uso de cualquier técnica.
Se ha estimado que las tablas del cual se usará la data será de:
\begin{itemize}
    \item sale\_order : Tabla de ordenes de ventas
    \item sale\_order\_line : Tabla de las líneas de ordenes de ventas
    \item res\_partner : Tabla de clientes
    \item crm\_lead : Tabla de las oportunidades e iniciativas
    \item crm\_phonecall : Tabla de las llamadas
\end{itemize}

Incluyendo las tablas intermedias.

\begin{thebibliography}{00}
\bibitem{b1} G. Eason, B. Noble, and I. N. Sneddon, ``On certain integrals of Lipschitz-Hankel type involving products of Bessel functions,'' Phil. Trans. Roy. Soc. London, vol. A247, pp. 529--551, April 1955.
\bibitem{b2} J. Clerk Maxwell, A Treatise on Electricity and Magnetism, 3rd ed., vol. 2. Oxford: Clarendon, 1892, pp.68--73.
\bibitem{b3} I. S. Jacobs and C. P. Bean, ``Fine particles, thin films and exchange anisotropy,'' in Magnetism, vol. III, G. T. Rado and H. Suhl, Eds. New York: Academic, 1963, pp. 271--350.
\bibitem{b4} K. Elissa, ``Title of paper if known,'' unpublished.
\bibitem{b5} R. Nicole, ``Title of paper with only first word capitalized,'' J. Name Stand. Abbrev., in press.
\bibitem{b6} Y. Yorozu, M. Hirano, K. Oka, and Y. Tagawa, ``Electron spectroscopy studies on magneto-optical media and plastic substrate interface,'' IEEE Transl. J. Magn. Japan, vol. 2, pp. 740--741, August 1987 [Digests 9th Annual Conf. Magnetics Japan, p. 301, 1982].
\bibitem{b7} M. Young, The Technical Writer's Handbook. Mill Valley, CA: University Science, 1989.
\end{thebibliography}
\vspace{12pt}
\color{red}
IEEE conference templates contain guidance text for composing and formatting conference papers. Please ensure that all template text is removed from your conference paper prior to submission to the conference. Failure to remove the template text from your paper may result in your paper not being published.

\end{document}
